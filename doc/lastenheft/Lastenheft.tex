\documentclass[12pt,a4paper]{article}
\usepackage[utf8]{inputenc}
\usepackage[german]{babel}
\usepackage{amsmath}
\usepackage{amsfonts}
\usepackage{amssymb}
\usepackage{makeidx}
\usepackage{graphicx}
\usepackage[left=2cm,right=2cm,top=2cm,bottom=2cm]{geometry}
\author{Katja Kaiser, Daniel Friedric}
\title{Lastenheft \\ SportControll}
\begin{document}

\maketitle



\section{Funktionale Anforderungen}
\begin{itemize}

\item* Erfassen persoenlicher Daten (Profile) Einstellungen
\item verschiedene Benutzer/Profile
\item Trainingseinheit abspeichern, aendern Loeschen(Datum, Ort, Sportart, Dauer, umfang, Intensitaet, bemerkung)
\item Liste Alletrainingseinheiten, Sortieren nach....
\item GUI 
	
	\item 	* gesamtuebersicht
 	\item  * Erfassungen Aenderung
	\item 	* persoeliche Einstellungen
\item  Statistik
	 \item	* Kalorienverbrauch
	 \item	* Durschnittstrainings Dauer
	 \item	* Wie viel welche Sportart
 	 
\item Trainingsplan, z.b. Ziele setzen?
\item export(ical, txt)
\item 
\end{itemize} 





\section{nicht Funktionale Anforderungen}
\begin{verbatim}
 * M-V-C-Pattern
 * Intuitive GUI
 * Einfach Ausfuehrbar
 * villeicht DB-Anbindung(SQLLite)
 * Swing
 * OO
 * Collectionklassen
 * JavaDoc
 * ausfuehrbares JAR
 * Exceptions
 
 
 
 == GUI ==
 
 === MenueBar ===
  * Datei
    * neues Profil
  	* Profil oeffnen
  	* speichern
  	* speichern unter
  	* export
  	* schliessen
  	
  * Profil
  	* Einstellungen
  	* Statistik
  	* Neues Profil
  	
  * Hilfe
  	* Hilfe

  	* ueber
  	
  	\end{verbatim}
\end{document} 